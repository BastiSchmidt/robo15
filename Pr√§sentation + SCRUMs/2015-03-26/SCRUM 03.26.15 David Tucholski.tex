\documentclass[ddcfooter,nototalpage]{tudbeamer}

\usepackage{german}
\usepackage[utf8]{inputenc}
\usepackage[T1]{fontenc}


\begin{document}
\einrichtung{Fakultät Mathematik und Naturwissenschaften}
\title{SCRUM - Tag 3}
\author{Team 29: Cherno Alpha \\ 
Friedrich Zahn \\ 
David Tucholski \\
Thomas Adlmaier \\ 
Sebastian Schmidt
}

\date{26.3.2015}
\maketitle

\section{Aktueller Stand}
\begin{frame}
\frametitle{Aktueller Stand}
\normalsize
\begin{itemize}
  \item Der Roboter kann das Labyrinth wie vorgegeben erkunden.
\end{itemize}
\end{frame}

\section{Tagesziele}
\begin{frame}
\frametitle{Tagesziele}
\normalsize
\begin{itemize}
  \item Zusätzliches Miteinbeziehen des Ultraschallsensors zur Vorzugsrichtungsoptimierung
\end{itemize}
\end{frame}

\section{Probleme}
\begin{frame}
\frametitle{Probleme}
\normalsize
\begin{itemize}
  \item Roboter fährt wegen der Motorumdrehungskorrekturschleifen ruckelig.
\end{itemize}
\end{frame}
\end{document}
%%% Local Variables:
%%% mode: latex
%%% TeX-master: t
%%% End:
