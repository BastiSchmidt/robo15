\documentclass[ddcfooter,nototalpage]{tudbeamer}
\usepackage{german}
\usepackage[utf8]{inputenc}
\usepackage[T1]{fontenc}
\begin{document}
\einrichtung{Fakultät Mathematik und Naturwissenschaften}
\title{SCRUM - Tag 4}
\author{Team 29: Cherno Alpha \\
Friedrich Zahn \\
David Tucholski \\
Thomas Adlmaier \\
Sebastian Schmidt
}
\date{19.3.2015}
\maketitle
\section{Aktueller Stand}
\begin{frame}
\frametitle{Aktueller Stand}
\normalsize

\textbf{Roboterteam:}
\begin{itemize}
\item Einige Bewegungsfunktionen fertiggestellt
\item Roboter kann zuverlässig einer Linie (einem Klebestreifen) folgen

\end{itemize}
\textbf{Algorithmusteam:}
\begin{itemize}
\item Datenstruktur fertiggestellt
\item Einige Hilfstools erstellt
\end{itemize}
\end{frame}
\section{Tagesziele}
\begin{frame}
\frametitle{Tagesziele}
\normalsize
\textbf{Roboterteam:}
\begin{itemize}

\item Kalibrierungsfunktionen für Lichtsensor und Motor
\item Funktion zum Knoten erkunden schreiben

\end{itemize}
\textbf{Algortithmusteam:}
\begin{itemize}
\item Wegfindungsalgorithmus schreiben
\end{itemize}
\end{frame}
\section{Probleme}
\begin{frame}
\frametitle{Probleme}
\normalsize
\textbf{Roboterteam}
\begin{itemize}
\item Ausgabe nach Knotenerkundung
\item Verschiedene Lichtsignale zuverlässig unterscheiden

\end{itemize}
\textbf{ }
\textbf{Algorithmusteam:}
\begin{itemize}
\item Debugging und Code-Optimierung
\end{itemize}
\end{frame}
\end{document}