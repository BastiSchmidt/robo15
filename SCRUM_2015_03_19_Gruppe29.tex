\documentclass[ddcfooter,nototalpage]{tudbeamer}
\usepackage{german}
\usepackage[utf8]{inputenc}
\usepackage[T1]{fontenc}
\begin{document}
\einrichtung{Fakultät Mathematik und Naturwissenschaften}
\title{SCRUM - Tag 4}
\author{Team 29: Cherno Alpha \\
Friedrich Zahn \\
David Tucholski \\
Thomas Adlmaier \\
Sebastian Schmidt
}
\date{19.3.2015}
\maketitle
\section{Aktueller Stand}
\begin{frame}
\frametitle{Aktueller Stand}
\normalsize
\begin{itemize}
\textbf{Roboterteam:}
\item Einige Bewegungsfunktionen fertiggestellt
\item Roboter kann zuverlässig einer Linie (einem Klebestreifen) folgen
\textbf{Algorithmusteam:}
\item Datenstruktur fertiggestellt
\item Einige Hilfstools erstellt
\end{itemize}
\end{frame}
\section{Tagesziele}
\begin{frame}
\frametitle{Tagesziele}
\normalsize
\begin{itemize}
\item Roboterteam: Umbau Roboter -> Lichtsensor weiter nach vorne  verlegen
\item Roboterteam: Kalibrierungsfunktionen für Lichtsensor und Motor
\item Roboterteam: Funktion zum Knoten erkunden schreiben
\item Algorithmusteam: Wegfindungsalgorithmus schreiben
\end{itemize}
\end{frame}
\section{Probleme}
\begin{frame}
\frametitle{Probleme}
\normalsize
\begin{itemize}
\item Roboterteam: Ausgabe nach Knotenerkundung
\item Roboterteam: Verschiedene Lichtsignale zuverlässig unterscheiden
\item Algorithmusteam: Debugging und Code-Optimierung
\end{itemize}
\end{frame}
\end{document}