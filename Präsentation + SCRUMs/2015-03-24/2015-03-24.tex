\documentclass[ddcfooter,nototalpage]{tudbeamer}

\usepackage{german}
\usepackage[utf8]{inputenc}
\usepackage[T1]{fontenc}


\begin{document}
\normalsize
\einrichtung{Fakultät Mathematik und Naturwissenschaften}
\title{SCRUM}
\author{Cherno Alpha\\Friedrich Zahn}
\maketitle

\section{Aktueller Stand}
\begin{frame}
\frametitle{Aktueller Stand}
Software:
\begin{itemize}
  \item Code erfolgreich zusammengeführt
  \item Speicherdesign an Roboterhardware angepasst
\end{itemize}
Hardware:
\begin{itemize}
  \item Bewegung und Scan funktionieren recht zuverlässig
  \item Auch längere Bewegungsabläufe (20+ Knoten) problemlos möglich
\end{itemize}
\end{frame}

\section{Tagesziele}
\begin{frame}
  \begin{itemize}
    \item Scanzuverlässigkeit verbessern
    \item Bewegungsgeschwindigkeit erhöhen
    \item Abschlussberichte erstellen
  \end{itemize}
\frametitle{Tagesziele}
\end{frame}

\section{Probleme}
\begin{frame}
\frametitle{Probleme}
\begin{itemize}
  \item Roboterverhalten und -zuverlässigkeit von Batteriestand und Umgebung 
  abhängig
  \item Codestruktur hat unter Zusammenführung gelitten, verringerte Les- und
  wartbarkeit (1200+ Zeilen)
\end{itemize}
\end{frame}
\end{document}