\documentclass[ddcfooter,nototalpage]{tudbeamer}

\usepackage{german}
\usepackage[utf8]{inputenc}
\usepackage[T1]{fontenc}


\begin{document}
\einrichtung{Fakultät Mathematik und Naturwissenschaften}
\title{SCRUM - Tag 3}
\author{Team 29: Cherno Alpha \\ Friedrich Zahn}
\date{18.3.2015}
\maketitle

\section{Aktueller Stand}
\begin{frame}
\frametitle{Aktueller Stand}
\begin{itemize}
  \item Schnittstellen zwischen Simulation und Hardware geklärt
  \item Motoren und Sensoren des Roboters lassen sich ansprechen
  \item Simulation stärker an Realität ausgerichtet
\end{itemize}
\end{frame}

\section{Tagesziele}
\begin{frame}
\frametitle{Tagesziele}
\begin{itemize}
  \item Gezielte Bewegung des Roboters und Knotenscan
  \item Entwurf des Algorithmus entsprechend Spezifikationen der Aufgabenstellung und Anpassung der Datenstruktur
  \item Erste Zusammenführung von Roboter- und Simulationscode
\end{itemize}
\end{frame}

\section{Probleme}
\begin{frame}
\frametitle{Probleme}
\begin{itemize}
  \item Aufgabenstellung stellt sehr spezifische Anforderungen und Optimierungserwartungen an Algorithmus
  \item Roboterhardware reagiert z.T. nicht erwartungsgemäß (Anfahr- und Abbremsverhalten)
  \item Software und toolchains beanspruchen weiterhin viel Zeit, die nicht für eigentliche Aufgabe genutzt werden kann
\end{itemize}
\end{frame}
\end{document}
%%% Local Variables:
%%% mode: latex
%%% TeX-master: t
%%% End:
